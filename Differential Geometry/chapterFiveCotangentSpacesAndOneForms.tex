\chapter{Cotangent Spaces and One Forms}
  \section{Linear algebra recap: The dual of a vector space}
    \label{sec: LA recap, dual of vector space}
    This section is merely a recap of linear algebra. For more information,
    one can refer to textbooks on linear algebra...

    Consider two vector spaces $V$ and $W$, and the set of all linear
    transformations from $V$ to $W$, denoted as $\mathcal{L}(V,W)$. On this
    set $\mathcal{L}(V,W)$, we define the binary addition operation as:
    \[(T_1 + T_2)(\vec{v}) = T_1(\vec{v}) + T_2(\vec{v})\]
    where $T_1,T_2 \in \mathcal{L}(V,W)$ and $\vec{v} \in V$.
    We also define the scalar multiplication operation as: 
    \[(cT)(\vec{v}) = cT(\vec{v})\] 
    where $T \in \mathcal{L}(V,W)$, $\vec{v} \in V$ and $c \in \mathbb{F}$
    where $\mathbb{F}$ is any field (e.g $\mathbb{R}$, $\mathbb{C}$).
    \begin{theorem}[$\mathcal{L}(V,W)$ with the binary addition and scalar
    multiplication defined above is a vector space over $\mathbb{F}$]
      \label{thm: lame dual space theorem}
    \end{theorem}
    \begin{proof}
      Left as an exercise, just verify that all the axioms of vector spaces
      are fulfilled. Note that the zero vector in $\mathcal{L}(V,W)$ is the
      zero map from $V$ to $W$, and the additive inverse of $T \in
      \mathcal{L}(V,W)$ is $-T \in \mathcal{L}(V,W)$, defined by $(-T)(\vec{v})
      = - T(\vec{v})$ where $\vec{v} \in V$.
    \end{proof}
    For the rest of this section, we shall assume that the field $\mathbb{F}$
    is $\mathbb{R}$. Now, we define the dual space of $V$ as $V^{*} \equiv
    \mathcal{L}(V, \mathbb{R})$\footnote{$\mathbb{R}$ has a vector space
    structure, so we can do this.}. From theorem~\ref{thm: lame dual space
    theorem}, $V^{*}$ has a vector space structure. We say that $V^{*}$ is
    the set of all linear functionals from $V$ to $\mathbb{R}$, i.e elements
    of $ V^{*} $ map $\vec{v} \in V$ to $\mathbb{R}$. 

    Suppose that $V$ has a basis $B = \{e_1, e_2, ... e_n\}$. Then, $V^*$ has
    a basis $B^* = \{f_1, f_2, ... f_n\}$ defined by: \[f_i(e_j) =
    \delta_{ij}\]
    
    Note that we can also write $V = \mathcal{L}(V^*, \mathbb{R})$, i.e we
    can consider $V$ the elements of $V$ as linear maps from $V^*$ to
    $\mathbb{R}$. In this case, we can write: \[e_i(f_j) = \delta_{ij}\]
    where $B = \{e_1, e_2, ... e_n\}$ is a basis of $V$ and $B^* = \{f_1,
    f_2, ... f_n\}$ is a basis of $V^*$. 

    To put both views on equal footing, we shall sometimes write: \[\langle
    v, k \rangle \equiv v(k) = k(v)\] where $v \in V$ and $k \in V^*$.
  \section{Cotangent Spaces}
    \label{sec: Cotangent Spaces}
    Now, we recall that $T_p(\mathcal{M})$ has a vector space structure.
    Thus, from section~\ref{sec: LA recap, dual of vector space}, we can
    define the another vector space dual to $T_p(\mathcal{M})$, which we
    shall denote as the cotangent space $T^*_p(\mathcal{M})$. Let's formalise this idea.
    \begin{definition}[Cotangent Space]
      The cotangent space at $p \in \mathcal{M}$ is defined as
      $T^*_p(\mathcal{M}) \equiv \mathcal{L}(T_p(\mathcal{M}), \mathbb{R})$.
    \end{definition}
    \begin{remark}
      I.e, $k \in T_p^*(\mathcal{M})$ is a real linear map from
      $T_p(\mathcal{M})$ into $\mathbb{R}$, defined by:
      \begin{align*}
        k: T_p(\mathcal{M}) &\rightarrow \mathbb{R} \\
        v &\mapsto \langle k, v \rangle_p
      \end{align*}
      where $v$ is any vector in $T_p(\mathcal{M})$, and the subscript $p$
      reminds us that all this happens only at a specific point $p \in
      \mathcal{M}$. 
    \end{remark}
    As per discussed in section~\ref{sec: LA recap, dual of vector space}, if
    $T_p(\mathcal{M})$ has a basis $\{e_1,...,e_m\}$, then the dual basis for
    $T^*_p(\mathcal{M})$, which we denote as $\{f^1,...,f^m\}$ can be
    uniquely determined by requiring that:
    \begin{equation}
      \label{eqn: dual basis defn}
      \langle f^i, e_j \rangle_p = \delta^i_j
    \end{equation}
    for all $i,j = 1,...,m$.
    \subsection{Note: positioning of indices}
      As the astute reader might have noticed, components of tangent vectors
      are labelled with upper indices, whereas basis vectors are labelled with
      lower indices. Similarly, components of cotangent vectors will be
      labelled with lower indices, and basis cotangent vectors will be labelled
      with upper indices. The position of the indices remind us of the
      transformation rules under a change of basis (i.e, when we move from one
      local chart to another). Things with upper indices transform
      contravariantly, like: \[v^{\prime i} = \frac{\partial x^{\prime
      i}}{\partial x^j} v^j\] whereas things with lower indices transform
      covariantly, like: \[k^{\prime}_i = \frac{\partial x^{j}}{\partial
      x^{\prime}_i} k_j \] It has been shown in the derivation of
      equation~\ref{eqn: Tangent vector component contravariant transformation}
      that the components of a tangent vector do indeed transform
      contravariantly, and it will be shown in subsection~\ref{subsec:
      transformation property of cotangent vectors under coord change} that
      components of a cotangent vector transform covariantly.
    \subsection{$T^*_p(\mathcal{M})$ in a local chart $(U, \phi)$}
      Recall that geometrically, tangent vectors are constructed from curves
      on a manifold. In a local chart $(U,\phi)$, a tangent vector
      $V^\sigma_p$ can be written as:
      \[V^\sigma_p \xrightarrow[\text{chart}]{\text{Local}} V^i
      \frac{\partial}{\partial x^i}\]
      Two important concepts to recall:
      \begin{enumerate}
        \item{$\{\frac{\partial}{\partial x^1},...,\frac{\partial}{\partial
          x^m}\}$ form a basis for $T_p(\mathcal{M})$ in a local chart. We
          can regard $\frac{\partial}{\partial x^i}$ as a directional
          derivative in the direction of increasing $x^i$.}
        \item{$V^\sigma_p$ in a local chart is just a linear combination of
          $\frac{\partial}{\partial x^i}$, where the components $V^i$ serve to
          produce a new directional derivative in the direction specified by
          $[\sigma]_p$.}
      \end{enumerate}
      Now, in a local chart, we want the dual basis of $T^*_p(\mathcal{M})$
      to be dual to $\{\frac{\partial}{\partial
      x^1},...,\frac{\partial}{\partial x^m}\}$, i.e we want to find a basis
      of $T^*_p(\mathcal{M})$ such that equation~\ref{eqn: dual basis defn}
      holds. Since in a local chart the $i$-th basis vector of
      $T_p(\mathcal{M})$ is a directional derivative in the direction of
      increasing $x^i$, a natural way to define the dual basis of
      $T^*_p(\mathcal{M})$ would be to define it as $\{dx^1,...,dx^m\}$, i.e
      we define the the $j$-th element of the dual basis to be a small change
      in the $x^j$ direction. The result is that we have: \[\left\langle
      \frac{\partial}{\partial x^i}, dx^j \right\rangle_{\phi(p)} =
      \frac{\partial}{\partial x^i}(dx^j) = \delta^j_i\] which automatically
      fulfils equation~\ref{eqn: dual basis defn}.
      
      Thus, for an arbitrary tangent vector $V^i \frac{\partial}{\partial
      x^i}$, we have:
      \[\left\langle V^i \frac{\partial}{\partial x^i}, dx^j
      \right\rangle_{\phi(p)} = V^j \]
      I.e, the basis cotangent vector $dx^j$ acting on an arbitrary tangent
      vector $V^i \frac{\partial}{\partial x^i}$ gives the length of that tangent
      vector in the $x^j$ direction.

      Anyway, since in a local chart $\{dx^1,...,dx^m\}$ is a basis for
      $T^*_p(\mathcal{M})$, we see that for $k \in T^*_p(\mathcal{M})$, $k$
      has local representation $k_i dx^i$.
  \section{The pull-back map between cotangent spaces}
    Recall from section~\ref{sec: push forward between tangent spaces} that
    under a mapping $\mathcal{F}: \mathcal{M} \rightarrow \mathcal{N}$ from a
    manifold $\mathcal{M}$ into a manifold $\mathcal{N}$, we could define the
    notion of a push-forward map between the tangent spaces of the two
    manifolds
    \[\mathcal{F}_* : T_p(\mathcal{M}) \rightarrow
    T_{\mathcal{F}(p)}(\mathcal{N})\]
    In the sense of cotangent spaces the map $\mathcal{F}$ induces a "pull-back" map between the two cotangent spaces:
    \[\mathcal{F}^* : T^*_{\mathcal{F}(p)}(\mathcal{N}) \rightarrow
    T^*_p(\mathcal{M})\]
    defined through
    \[\left\langle \mathcal{F}^*k, v \right\rangle_p = \left\langle
    k, \mathcal{F}_*v \right\rangle_{\mathcal{F}(p)}\] for all $k \in
    T^*_{\mathcal{F}(p)}(N)$ and $v \in T_p(\mathcal{M})$. Now, at this
    juncture we shall introduce a theorem regarding this pull-back map.
    \begin{theorem}[Composition of pull-back maps]
      Suppose that $\mathcal{M}$,$\mathcal{N}$,$\mathcal{P}$ are three differentiable manifolds with differentiable maps:
      \[\mathcal{M} \xrightarrow{\mathcal{F}_1} \mathcal{N}
      \xrightarrow{\mathcal{F}_2} \mathcal{P}\]
      Then, we have:
      \[\left(\mathcal{F}_2 \circ \mathcal{F}_1\right)^* = \mathcal{F}_1^*
      \circ \mathcal{F}_2^*\]
    \end{theorem}
    \begin{proof}[Rough sketch of a proof]
      First, we can easily prove that $\left( \mathcal{F}_2 \circ
      \mathcal{F}_1\right)_*$ = ${\mathcal{F}_2}_* \circ {\mathcal{F}_1}_*$.
      Then, we just use the definition of the pull-back map twice.
    \end{proof}

    The pull-back map defined above is very useful, and allows us to determine many properties of cotangent vectors based on the properties of tangent vectors. We shall see two applications below.

    \subsection{Application one: Constructing the basis for
    $T^*_p(\mathcal{M})$}
      Recall from subsection~\ref{subsec: push forward map induced by a homeomorphism} that if $\{\frac{\partial}{\partial
      x^1},...,\frac{\partial}{\partial x^1}\}$ is a basis of
      $T_p(\mathcal{M})$ in a local chart $(U, \phi)$, then the push forward
      map $\phi^{-1}_*$ allows us to determine the basis $\{e_1,...,e_m\}$ of
      $T_p(\mathcal{M})$.
      We can construct the basis for $T^*_p(\mathcal{M})$ in a similar way. We have: 
      \begin{align*}
        \left\langle \phi^* dx^i, e_j \right\rangle_p 
        &= \left\langle dx^i, \phi_*e_j \right\rangle_{\phi(p)} \\
        &= \left\langle dx^i, \frac{\partial}{\partial x^j} \right\rangle_{\phi(p)} \\
        &= \delta^i_j
      \end{align*}
      which tells if we define $f^i = \phi^* dx^i$, then $\{f^1,...,f^m\}$
      forms the dual basis for $T^{*}_p(\mathcal{M})$.
    \subsection{Application two: Transformation property of cotangent vectors under a change of coords}
      \label{subsec: transformation property of cotangent vectors under coord change}
      Suppose that $(U,\phi)$ and $(V,\psi)$ are two charts of a manifold
      $\mathcal{M}$. Consider a point $p \in U \cap V$. Let $k \in
      T^*_p(\mathcal{M})$ and $v \in T_p(\mathcal{M})$. Also, let $\bar{k},
      \bar{v}$ and $\bar{k}^\prime, \bar{v}^\prime$ be the coordinate
      representations of $k, v$ in the local charts $(U,\phi)$, $(V,\psi)$
      respectively. Aim: determine how the components of $\bar{k}^\prime$ are
      related to the components of $\bar{k}$.\\ Since we have:
      \begin{align*}
        \left\langle \bar{k}, \bar{v}\right\rangle_{\phi(p)}
        &= \left\langle \bar{k}, \phi_{*} v\right\rangle_{\phi(p)} \\
        &= \left\langle \phi^{*}\bar{k},  v\right\rangle_{p} \\
        &= \left\langle k,  v\right\rangle_{p}
      \end{align*}
      and we can similarly show:
        \[\left\langle \bar{k}^\prime, \bar{v}^\prime\right\rangle_{\psi(p)}
        = \left\langle k, v\right\rangle_{p}\]
      we arrive at this result:
        \begin{equation}
          \label{eqn: contraction coordinate independence}
          \left\langle \bar{k}^\prime, \bar{v}^\prime\right\rangle_{\psi(p)}
          = \left\langle \bar{k}, \bar{v}\right\rangle_{\phi(p)}
        \end{equation}
      First, we evaluate the LHS of equation~\ref{eqn: contraction coordinate
      independence} to give us:
      \begin{align}
        \left\langle \bar{k}^\prime, \bar{v}^\prime\right\rangle_{\psi(p)} 
        &= \bar{k}_{i}^{\prime} \bar{v}^{\prime j} \left\langle dx^{\prime i} ,
        \frac{\partial}{\partial x^{\prime j}} \right\rangle \nonumber \\
        &= \bar{k}_{i}^{\prime} \bar{v}^{\prime j} \delta^i_j \nonumber \\
        &= \bar{k}_{i}^{\prime} \bar{v}^{\prime i} \label{eqn:
        covector change of coord part 1}
      \end{align}
      Then, we can simiarly evaluate the RHS of equation~\ref{eqn: contraction coordinate independence} to give us:
      \begin{equation}
        \label{eqn: covector change of coord eqn part 2}
        \left\langle \bar{k}, \bar{v}\right\rangle_{\phi(p)} = \bar{k}_j
        \bar{v}^j
      \end{equation}
      Now, we know that $\overline{\mathcal{F}_*}\bar{v} = \bar{v}^\prime$,
      and we also know that $\bar{v}^{\prime i} = \frac{\partial x^{\prime
      i}}{\partial x^j} \bar{v}^j$. Thus, using this as well as
      equations~\ref{eqn: covector change of coord part 1}, \ref{eqn:
      covector change of coord eqn part 2} we have:
      \begin{gather}
        \left\langle \bar{k}^\prime, \bar{v}^\prime\right\rangle_{\psi(p)} =
        \left\langle \bar{k}, \bar{v}\right\rangle_{\phi(p)} \nonumber\\
        \implies \bar{k}_{i}^{\prime} \bar{v}^{\prime i} = \bar{k}_j
        \bar{v}^j \nonumber\\
        \implies \bar{k}_{i}^{\prime} \left(\frac{\partial x^{\prime
        i}}{\partial x^j} \bar{v}^j \right) = \bar{k}_j \bar{v}^j \nonumber\\
        \implies \bar{k}_{i}^{\prime} \left(\frac{\partial x^{\prime
        i}}{\partial x^j} \right) = \bar{k}_j \label{eqn: covector change of coord eqn part 3}
      \end{gather}
      We can then easily invert\footnote{We do this inversion either by
      swapping the primed and unprimed variables, or we write the equation in
      matrix form and then invert. When we do the latter approach, we realise
      that $\left(\frac{\partial x^j}{\partial x^{\prime i}} \right)$ is
      nothing more than just entries of the Jacobian matrix of the coordinate
      transform.} equation~\ref{eqn: covector change of coord eqn part 3} to
      give us:
      \begin{equation}
        \bar{k}_{i}^{\prime} = \left(\frac{\partial x^j}{\partial x^{\prime
        i}} \right) \bar{k}_j
      \end{equation}
      which tells us how the components of a cotangent vector transform under
      coordinate transformation. As can be seen, the components transform
      covariantly.
  \section{One-forms}
    \begin{definition}[1-form]
      \label{defn: one form defn}
      A one-form $\omega$ on $\mathcal{M}$ is a smooth assignment of a
      cotangent vector $\omega_p \in T_p^*(\mathcal{M})$ to each point $p \in
      \mathcal{M}$. Here, "smooth" means that for any vector field $X \in \mathcal{X}(\mathcal{M})$, the real-valued function
        \[ \left\langle\omega, \mathcal{X}\right\rangle(p) = \left\langle
        \omega_p, X_p \right\rangle\]
      is smooth.
    \end{definition}
    \begin{remark}
      To better understand the notion of smoothness, we first go to a local
      coordinate chart $(U, \phi)$. We note that:
      \begin{align*}
        \left\langle \omega_p, X_p \right\rangle_p
        &= \left\langle \phi^* \bar{\omega}_p, X_p \right\rangle_p \\
        &= \left\langle  \bar{\omega}_p, \phi_* X_p \right\rangle_{\phi(p)}\\
        &= \left\langle  \bar{\omega}_p, \bar{X}_p \right\rangle_{x}
      \end{align*}
      where in the last line, we have defined $x = \phi(p)$. Now, since
      $\bar{X}_p = \bar{X}^i(x) \frac{\partial}{\partial x^i}$ and
      $\bar{\omega}_p = \bar{\omega}_j(x) dx^j$, we have:
      \begin{equation}
        \label{eqn: one form smoothness defn part 1}
        \left\langle \omega_p, X_p \right\rangle_p = \bar{\omega}_i(x)
        \bar{X}^i(x)
      \end{equation}
      Thus, "smooth assignment of a cotangent vector $\omega_p$" in
      definition~\ref{defn: one form defn} just means that for any chart
      $(U,\phi)$ of $\mathcal{M}$, and for any $X \in
      \mathcal{X}(\mathcal{M})$, the expression $\bar{\omega}(x)_i
      \bar{X}^i(x)$ is a smooth function of $x$.

      In fact, since $X^i(x)$ is by definition already a smooth function of $x$, we just require $\omega_i(x)$ to be a smooth function of $x$.

      \paragraph{TL:DR} Let $(U, \phi)$ is an arbitrary chart of the
      manifold, and let $x = \phi(p)$ for arbitrary $p \in \mathcal{M}$. A
      one-form $\omega$ has a local coord representation of $\omega_i(x)
      dx^i$. The function $\omega_i(x)$ is a smooth function of $x$.
      \subsection{The pull-back of a one-form}
        The pull-back of a one-form is very similar to the pull-back of
        cotangent vector. Just note that since a one-form is a smooth
        assignment of cotangent vectors to all points in the manifold, we have to pull-back at every point in the manifold.
        \begin{definition}[Pull-back of a 1-form]
          Let $\mathcal{F}: \mathcal{M} \rightarrow \mathcal{N}$ be a
          differentiable map\footnote{I'm not sure if differentiable map is good enough...do we actually require a diffeomorphism? I'm not sure.} between two manifolds $\mathcal{M}$ and
          $\mathcal{N}$. If $\omega$ is a one-form on $\mathcal{N}$ then the
          pull-back of $\omega$ is the one-form $\mathcal{F}^*\omega$ on
          $\mathcal{M}$ defined by:
            \[\left\langle \mathcal{F}^*\omega, v \right\rangle_p =
            \left\langle \omega, \mathcal{F}_*v
            \right\rangle_{\mathcal{F}(p)} \]
          for all points $p \in \mathcal{M}$ and all tangent vectors $v \in T_p(\mathcal{M})$.
        \end{definition}
    \end{remark}

      